\documentclass[12pt]{article}
\usepackage{amsmath,amssymb,amsthm,verbatim}

%*****************************************************************
%			Dimensions

\setlength{\textheight}{9in}
\setlength{\textwidth}{6.5in}
\setlength{\oddsidemargin}{0in}
\setlength{\evensidemargin}{0in}
\setlength{\hoffset}{0in}
\setlength{\voffset}{-1in}
\setlength{\footskip}{.5in}
\setlength{\parskip}{10pt}
\setlength{\parindent}{0pt}

\pagestyle{empty}

%*****************************************************************

%****************************************************************
                

\begin{document}

\begin{flushright}
\begin{tabular}{l}
Christopher David Miller \\  % Enter your name here
FSMAT 201 \\  % The current class. You can change this if you want to use this for other classes.
Problem 3.1 \\  % Replace x with the appropriate problem number
Version 1 \\ %If this is the first time you've submitted this solution, let n=1. If this is a first resubmission, let n=2, etc. 
\end{tabular}
\end{flushright}
\vspace{20pt}  % Just adding some extra vertical space.

% Now write out the statement of the problem, followed by your solution. 

\textbf{3.1} \\  % Replace x with the appropriate problem number
Let $z = a + bi$ and $w = c + di $. Prove that $\overline{zw} = \overline{z} \cdot \overline{w}$.



Now we have enough definitions to finish our proof.
\begin{proof}
  To prove this statement we need to introduce the definition of the conjugate of a complex number. Let $q$ be a complex number, in other words $q = c + di$ where $c$ and $d$ are real numbers and where $i = \sqrt{-1}.$ The conjugate of $q$, denoted by $\overline{q}$, is defined by $$\overline{q}= c - di.$$
  Let $z = a + bi$ and $w = c + di $. In order to show that  $\overline{zw} = \overline{z} \cdot \overline{w}$. We need to calculate $\overline{zw}$ and $\overline{z} \cdot \overline{w}$ .
  $$
  \begin{aligned}
    \overline{zw} &= \overline{(a+bi)(c+di)} \\ 
     &= \overline{ac+adi+bci-bd}\\
     &= \overline{(ac-bd)+i(ad+bc)}\\
     &= (ac-bd)-i(ad+bc)  
  \end{aligned}
  $$
  \begin{equation*}
    \begin{aligned}
      \overline{z} \cdot \overline{w} &= \overline{(a+bi)} \cdot \overline{(c+di)} \\ 
       &= (a-bi) \cdot (c-di)\\
       &= ac -adi-bci-bd\\
       &= (ac-bd)-i(ad+bc)  
    \end{aligned}
  \end{equation*}
  We have shown that $\overline{zw} = \overline{z} \cdot \overline{w}$
    
      
      

    
\end{proof}
\newpage
\begin{flushright}
  \begin{tabular}{l}
  Christopher David Miller \\  % Enter your name here
  FSMAT 201 \\  % The current class. You can change this if you want to use this for other classes.
  Problem 3.2 \\  % Replace x with the appropriate problem number
  Version 1 \\ %If this is the first time you've submitted this solution, let n=1. If this is a first resubmission, let n=2, etc. 
  \end{tabular}
  \end{flushright}
  \vspace{20pt}  % Just adding some extra vertical space.
  
  % Now write out the statement of the problem, followed by your solution. 
  
  \textbf{3.2} \\  % Replace x with the appropriate problem number
  Let $z = a + bi$ and $w = c + di $. Prove that $\lvert zw \rvert = \lvert z \rvert \cdot \lvert w \rvert$.
  
  \begin{proof}
    To prove this statement we need to introduce the definition of the modulus of a complex number. Let $q$ be a complex number, in other words $q = c + di$ where $c$ and $d$ are real numbers. The modulus of $q$, denoted by $\lvert q \rvert$, is defined by $$\lvert q \rvert= \sqrt{c^2 + d^2}$$
    Let $z = a + bi$ and $w = c + di $. In order to show that  $\lvert zw \rvert = \lvert z \rvert \cdot \lvert w \rvert$. We need to calculate $\lvert zw \rvert$ and $\lvert z \rvert \cdot \lvert w \rvert.$
    $$\begin{aligned}
      \lvert z \rvert \cdot \lvert w \rvert &= \sqrt{a^2 + b^2} \cdot \sqrt{c^2 + d^2} \\ 
       &= \sqrt{(a^2 + b^2)(c^2 + d^2)}\\
       &= \sqrt{a^2c^2 + a^2d^2 + b^2c^2+b^2d^2}
    \end{aligned}$$
    \begin{equation*}
      \begin{aligned}[c]
        \lvert zw \rvert &= \overline{(a+bi)} \cdot \overline{(c+di)} \\ 
         &= (a-bi) \cdot (c-di)\\
         &= ac -adi-bci-bd\\
         &= (ac-bd)-i(ad+bc)  
      \end{aligned}
    \end{equation*}
    We have shown that $\overline{zw} = \overline{z} \cdot \overline{w}$
      
        
        
  
      
  \end{proof}
  \newpage
\begin{flushright}
  \begin{tabular}{l}
  Christopher David Miller \\  % Enter your name here
  FSMAT 201 \\  % The current class. You can change this if you want to use this for other classes.
  Problem 3.3 \\  % Replace x with the appropriate problem number
  Version 1 \\ %If this is the first time you've submitted this solution, let n=1. If this is a first resubmission, let n=2, etc. 
  \end{tabular}
  \end{flushright}
  \vspace{20pt}  % Just adding some extra vertical space.
  
  % Now write out the statement of the problem, followed by your solution. 
  
  \textbf{3.3} \\  % Replace x with the appropriate problem number
  Let $z = a + bi$. Find the real and imaginary parts of $\frac{1}{z}.$
  
  \textbf{Solution:}
      In order to find the real and imaginary parts of a complex number we must express $\frac{1}{z}$ of the form $c+di$ where $c$ and $d$ are real numbers.First we have 
      $$
        \frac{1}{z}  = \frac{1}{a+bi},
      $$
      then we multiply by 1 where $1 = \frac{a-bi}{a-bi},$
      \begin{align*}
        &=\frac{1}{a+bi} \cdot \frac{a-bi}{a-bi}\\
        &=\frac{a-bi}{a^2+b^2} \\
        &=\frac{a}{a^2+b^2}-\frac{bi}{a^2+b^2} \\
        &=\frac{a}{a^2+b^2}-i\frac{b}{a^2+b^2} \\
      \end{align*}.
      Now our complex number is of the proper form. So our real part is $\frac{a}{a^2+b^2}$ and the imaginary part is $-\frac{b}{a^2+b^2}$.
      
      
      \newpage
      \begin{flushright}
        \begin{tabular}{l}
        Christopher David Miller \\  % Enter your name here
        FSMAT 201 \\  % The current class. You can change this if you want to use this for other classes.
        Problem 3.4 \\  % Replace x with the appropriate problem number
        Version 1 \\ %If this is the first time you've submitted this solution, let n=1. If this is a first resubmission, let n=2, etc. 
        \end{tabular}
        \end{flushright}
        \vspace{20pt}  % Just adding some extra vertical space.
        
        % Now write out the statement of the problem, followed by your solution. 
        
        \textbf{3.4} \\  % Replace x with the appropriate problem number
        Let $z = a + bi$. Prove that z is a real number if and only if $z = \overline{z}$
        
        \begin{proof}
          To prove this statement we need to introduce the definition of the conjugate of a complex number. Let $q$ be a complex number, in other words $q = c + di$ where $c$,the real part, and $d$, the imaginary part, are real numbers and where $i = \sqrt{-1}.$ The conjugate of $q$, denoted by $\overline{q}$, is defined by $$\overline{q}= c - di.$$
          Let $z = a + bi$. Assume $z$ is a real number.Thus z does not have an imaginary part or in other words $z = a +0i$. Notice from the definition of conjugate that $\overline{z} = a-0i$. Thus $z = \overline{z}$. Now assume that $z = \overline{
            z
          }.$ We will show that $z$ is a real number. For $z$ to be a real number then the imaginary part of $z$ must be 0.Notice since 
          $z = \overline{
            z
          }.$ then $a+bi = a - bi$ the only real number $b$ that this is true for is 0. Thus $z$ is a real number.  
        \end{proof}   
  
      
        \newpage
      \begin{flushright}
        \begin{tabular}{l}
        Christopher David Miller \\  % Enter your name here
        FSMAT 201 \\  % The current class. You can change this if you want to use this for other classes.
        Problem 3.5 \\  % Replace x with the appropriate problem number
        Version 1 \\ %If this is the first time you've submitted this solution, let n=1. If this is a first resubmission, let n=2, etc. 
        \end{tabular}
        \end{flushright}
        \vspace{20pt}  % Just adding some extra vertical space.
        
        % Now write out the statement of the problem, followed by your solution. 
        
        \textbf{3.5} \\  % Replace x with the appropriate problem number
         Prove DeMoivre's Theorem, namely, that for a nonnegative integer n, $$ (\cos\omega + i\sin\omega)^n = \cos n\omega+i\sin n\omega $$
        
        \begin{proof}
          To prove this statement we need to introduce the definition of the exponetial function. Let $q$ be a complex number, in other words $q = c + di$ where $c$,the real part, and $d$, the imaginary part, are real numbers and where $i = \sqrt{-1},$ then we define the exponetial function $e^q$ by $$ e^q = e^c(\cos d +i\sin d ) .$$Notice if the real part of $q=0$ then $e^q = (\cos d +i\sin d )$  
          
          Assume $(\cos\omega + i\sin\omega)^n$, then following from above we have 
          \begin{align*}
            (\cos\omega + i\sin\omega)^n = (e^{i\omega})^n
          \end{align*}
          Following from laws of exponets we have
          \begin{align*}
            (e^{i\omega})^n = (e^{i\omega n}).
          \end{align*}
          Then using our defintion from earlier we have,
          \begin{align*}
            (e^{i\omega n})= \cos n\omega+i\sin n\omega.
          \end{align*}
          The proof for the other way is similar.
        \end{proof}  






\end{document}











