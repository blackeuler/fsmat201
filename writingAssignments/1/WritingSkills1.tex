\documentclass[12pt]{article}
\usepackage{amsmath,amssymb,amsthm,verbatim}

%*****************************************************************
%			Dimensions

\setlength{\textheight}{9in}
\setlength{\textwidth}{6.5in}
\setlength{\oddsidemargin}{0in}
\setlength{\evensidemargin}{0in}
\setlength{\hoffset}{0in}
\setlength{\voffset}{-1in}
\setlength{\footskip}{.5in}
\setlength{\parskip}{10pt}
\setlength{\parindent}{0pt}




%*****************************************************************

%****************************************************************
    
\pagestyle{empty}              

\begin{document}
\begin{center}
\textbf{Writing Skills 1\\
Basic \LaTeX Commands}
\end{center}

\begin{enumerate}
    \item If $s_n = a + ar + ar^2 + \cdots + ar^{n-1}$, then 
        \[s_n = \sum_{k=1}^{n} ar^{k-1} = \frac{a(1-r^n)}{1-r}.\]
    \item Here are some useful facts from Calculus:
        \begin{enumerate}
            \item $\displaystyle \frac{d}{dx}(\ln x) = \frac{1}{x}$
            \item Double angle formulas
                \begin{itemize}
                    \item $\sin(2x) = 2\sin x \cos x$
                    \item $\cos(2x) = \cos^2x - \sin^2x$
                \end{itemize}
            \item If $n \neq -1$, then  $\displaystyle \int x^n dx = \frac{x^{n+1}}{n+1} +c$
        \end{enumerate}
    \item If $(x_1,y_1)$ and $(x_2,y_2)$ are two points in the plane with $x_1 \neq x_2$, then the slope of the line through these points is
    \begin{equation}
        \frac{y_2 - y_1}{x_2 - x_1}
        .
    \end{equation}
    We sometimes denote the slope given in Equation (1) by m.
    \item If \[ x_n = \frac{n}{n+1}  \text{ and}  i_k = 2k+1 ,\] then
                \[x_{i_{12}} = \frac{i_{12}}{i_{12}+1}= \frac{25}{26}.\]
    \item Completing the sqaure, we see that if $y = 2x^2 + 8x +7,$ then 
        \begin{align*}
            y &= 2(x^2 +4x)+7 \\
            &= 2(x^2+4x+4)+7-2(4)\\
            &=2(x+2)^2 -1. 
        \end{align*}
        
        
    
\end{enumerate}

\end{document}



