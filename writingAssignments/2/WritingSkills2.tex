\documentclass[12pt]{article}
\usepackage{amsmath,amssymb,amsthm,verbatim}

%*****************************************************************
%			Dimensions

\setlength{\textheight}{9in}
\setlength{\textwidth}{6.5in}
\setlength{\oddsidemargin}{0in}
\setlength{\evensidemargin}{0in}
\setlength{\hoffset}{0in}
\setlength{\voffset}{-1in}
\setlength{\footskip}{.5in}
\setlength{\parskip}{10pt}
\setlength{\parindent}{0pt}




%*****************************************************************

%****************************************************************
              
\pagestyle{empty}    

\begin{document}

\begin{center}
    \textbf{Writing Skills 2\\
        Typesetting Matrices and Vectors}
\end{center}

An $m \times n$ matrix is an array of $m$ rows and $n$ columns. The notation $a_{ij}$
is used to denote the entry in the $i$th row and the $j$th column of a matrix $A$, so a general $m \times n$ matrix $A$ has the form
$$ \begin{bmatrix}
        a_{11} & a_{12} & \cdots & a_{1n} \\
        a_{21} & a_{22} & \cdots & a_{2n} \\
        \vdots & \vdots & \ddots & \vdots \\
        a_{m1} & a_{m2} & \cdots & a_{mn}
    \end{bmatrix}    .$$
A vector $ \mathbf{u} \in \mathbb{R}^n$ is an $n \times 1$ matrix. Thus, $\mathbf{u}$ has the form $$ \mathbf{u} = \begin{bmatrix}
        u_1 \\u_2\\\vdots\\u_n\\
    \end{bmatrix},$$
where $u_1,u_2, \ldots , u_n$ are real numbers. To save space, it is often convenient to express a vector in terms of its transpose, so $\mathbf{u} = [u_1,u_2,\ldots,u_n]^T$ or $ \mathbf{u}^T = [u_1,u_2,\ldots,u_n]$.

Given $\mathbf{u} = [u_1,u_2,\ldots,u_n]^T$ and $\mathbf{v} = [v_1,v_2,\ldots,v_n]^T$, the inner product of $\mathbf{u}$ and $\mathbf{v}$, $\mathbf{u} \cdot \mathbf{v}$, is defined by
\[ \mathbf{u} \cdot \mathbf{v} = \sum_{k=1}^{n} u_k v_k = u_1v_1 + u_2v_2 + \cdots + u_n v_n .\]
The $norm$ of  a vector $\mathbf{u}$ is denoted by $\|u\|$ and defined by \[ \|u\| = \sqrt{ \sum_{k=1}^{n} u_k^2} = \sqrt{u_1^2+u_2^2+\cdots + u_n^2} .\]
Note that $\|\mathbf{u}\| = \sqrt{\mathbf{u} \cdot \mathbf{u}}.$

The matrix $$ \begin{bmatrix}
        \phantom{-}3 & \phantom{-}0 & -1           & \phantom{-}5 & \phantom{-}9 & -2           \\
        -5           & \phantom{-}2 & \phantom{-}4 & \phantom{-}0 & -3           & \phantom{-}1 \\
        -8           & -6           & \phantom{-}3 & \phantom{-}1 & \phantom{-}7 & -4
    \end{bmatrix}$$
can also be written, for example, as the $2 \times 3$ partitioned matrix or block matrix
$$ \left[
        \begin{array}{ccc|cc|c}
            \phantom{-}3& \phantom{-}0& -1 &\phantom{-}5 &\phantom{-}9 &-2 \\
            -5& \phantom{-}2& \phantom{-}4& \phantom{-}0 &-3 &\phantom{-}1\\
            \hline
            -8& -6& \phantom{-}3& \phantom{-}1 &\phantom{-}7 &-4
        \end{array}
        \right]
$$
\end{document}



